\chapter{\textit{$\Pi$ Autômato}}\label{cap2}
Esse capítulo tratará do $\Pi$ Autômato antes da introdução das continuações, portanto, do seu estado atual.
\section{Definição}
O $\Pi$ Autômato é uma máquina abstrata, com seu estado contendo duas pilhas (a saber, pilha de valores, que armazena principalmente valores intermediários de uma computação; e pilha de controle, que armazena o resto dos comandos e/ou declarações a partir do ponto atual de execução) e dois mapas finitos (a saber, o "ambiente", que controla os bindings de constantes e a associação entre identificadores e índices (chamados "localizações") do segundo hash, chamado de "memória", que mapeia as localizações para números, booleanos e abstrações. Este autômato admite transições de estados através de regras funcionais, dirigidas por sintaxe.
\subsection{PI IR}
O $\Pi$ Ir é um conjunto de estruturas utilizadas para descrever elementos semânticos. Essas estruturas se dividem em expressões, comandos e declarações, dependendo do seu efeito quando computados. Expressões, ao serem computadas, produzem um valor. Esse valor pode ser um número ou um valor booleano. Nenhuma mudança na memória ou nos \texttt{bindings} é permitida. Comandos, ao serem computados, podem produzir mudanças na memória, porém não nos \texttt{bindings}. Comandos não produzem valores. As declarações produzem mudanças nos \texttt{bindings} e não produzem valores.
\subsubsection{Expressões}
As expressões são divididas em expressões aritméticas e expressões booleanas. Expressões aritméticas são aquelas que, ao serem computadas, produzem um número. Expressões booleanas são aquelas que, ao serem computadas, produzem um valor booleano (\texttt{true} ou \texttt{false}). As expressões são as seguintes : 

\begin{table}[]
    \centering
    \begin{tabular}{|c|c|c|}
    \hline
         Nome da expressão & Aridade & Categoria\\\hline\hline
         add & 2 & Aritmética\\\hline
         sub & 2 & Aritmética\\\hline
         mul & 2 & Aritmética\\\hline
         div & 2 & Aritmética\\\hline
         num & 1 & Aritmética\\\hline
         bool & 1 & Booleana\\\hline
         or & 2 & Booleana\\\hline
         and & 2 & Booleana\\\hline
         ge & 2 & Booleana\\\hline
         gt & 2 & Booleana\\\hline
         lt & 2 & Booleana\\\hline
         neg & 1 & Booleana\\\hline
         eq & 2 & Booleana\\\hline
         le & 2 & Booleana\\\hline
    \hline
    \end{tabular}
    \caption{Conjunto de expressões descritas nesse trabalho}
    \label{tab:expressoes}
\end{table}
\begin{table}[]
    \centering
    \begin{tabular}{|c|c|c|}
    \hline
         Nome da expressão & Aridade & Categoria\\\hline\hline
         add & 2 & Aritmética\\\hline
         sub & 2 & Aritmética\\\hline
         mul & 2 & Aritmética\\\hline
         div & 2 & Aritmética\\\hline
         num & 1 & Aritmética\\\hline
         bool & 1 & Booleana\\\hline
         or & 2 & Booleana\\\hline
         and & 2 & Booleana\\\hline
         ge & 2 & Booleana\\\hline
         gt & 2 & Booleana\\\hline
         lt & 2 & Booleana\\\hline
         neg & 1 & Booleana\\\hline
         eq & 2 & Booleana\\\hline
         le & 2 & Booleana\\\hline
    \hline
    \end{tabular}
    \caption{Significado intuitivo das expressões}
    \label{tab:significadoExpressoes}
\end{table}
TRANSIÇÕES NO PI-AUTOMATO
\subsubsection{Comandos}
\subsubsection{Declarações}

\subsection{Exemplos de PI IR}
\subsubsection{Expressões}
\subsubsection{Comandos}
\subsubsection{Declarações}\label{declarações}