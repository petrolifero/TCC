\chapter{\textit{$\Pi$ Autômato}}\label{cap2}
Esse capítulo tratará do $\Pi$ Autômato antes da introdução das continuações, portanto, do seu estado atual.
\section{Definição}
O $\Pi$ Autômato é uma máquina abstrata, com seu estado contendo duas pilhas (a saber, pilha de valores, que armazena principalmente valores intermediários de uma computação; e pilha de controle, que armazena o resto dos comandos e/ou declarações a partir do ponto atual de execução) e dois mapas finitos (a saber, o "ambiente", que controla os bindings de constantes e a associação entre identificadores e índices (chamados "localizações") do segundo hash, chamado de "memória", que mapeia as localizações para números, booleanos e abstrações. Este autômato admite transições de estados através de regras funcionais, dirigidas por sintaxe.
\subsection{PI IR}

\subsubsection{Expressões}
\subsubsection{Comandos}
\subsubsection{Declarações}

\subsection{Exemplos de PI IR}
\subsubsection{Expressões}
\subsubsection{Comandos}
\subsubsection{Declarações}\label{declarações}