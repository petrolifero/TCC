%% abtex2-modelo-trabalho-academico.tex, v-1.9.3 laurocesar
%% Copyright 2012-2015 by abnTeX2 group at http://abntex2.googlecode.com/ 
%% This work may be distributed and/or modified under the
%% conditions of the LaTeX Project Public License, either version 1.3
%% of this license or (at your option) any later version.
%% The latest version of this license is in
%%   http://www.latex-project.org/lppl.txt
%% and version 1.3 or later is part of all distributions of LaTeX
%% version 2005/12/01 or later.
%%
%% This work has the LPPL maintenance status `maintained'.
%% 
%% The Current Maintainer of this work is the abnTeX2 team, led
%% by Lauro César Araujo. Further information are available on 
%% http://abntex2.googlecode.com/
%%
%% This work consists of the files abntex2-modelo-trabalho-academico.tex,
%% abntex2-modelo-include-comandos and abntex2-modelo-references.bib
%%

% ------------------------------------------------------------------------
% ------------------------------------------------------------------------
% abnTeX2: Modelo de Trabalho Academico (tese de doutorado, dissertacao de
% mestrado e trabalhos monograficos em geral) em conformidade com 
% ABNT NBR 14724:2011: Informacao e documentacao - Trabalhos academicos -
% Apresentacao
% ------------------------------------------------------------------------
% ------------------------------------------------------------------------

\documentclass[
	% -- opções da classe memoir --
	12pt,				% tamanho da fonte
	openright,			% capítulos começam em pág ímpar (insere página vazia caso preciso)
	twoside,			% para impressão em verso e anverso. Oposto a oneside
	a4paper,			% tamanho do papel. 
	% -- opções da classe abntex2 --
	%chapter=TITLE,		% títulos de capítulos convertidos em letras maiúsculas
	%section=TITLE,		% títulos de seções convertidos em letras maiúsculas
	%subsection=TITLE,	% títulos de subseções convertidos em letras maiúsculas
	%subsubsection=TITLE,% títulos de subsubseções convertidos em letras maiúsculas
	% -- opções do pacote babel --https://www.overleaf.com/project/5da65c39dd4248000106886a
	english,			% idioma adicional para hifenização
	brazil				% o último idioma é o principal do documento
	]{abntex2}

% ---
% Pacotes básicos 
% ---
%\usepackage{uarial}				% Usa a fonte Latin Modern			lmodern
\usepackage[T1]{fontenc}		% Selecao de codigos de fonte.
\usepackage[utf8]{inputenc}		% Codificacao do documento (conversão automática dos acentos)
\usepackage{lastpage}			% Usado pela Ficha catalográfica
\usepackage{indentfirst}		% Indenta o primeiro parágrafo de cada seção.
\usepackage{color}				% Controle das cores
\usepackage{graphicx}			% Inclusão de gráficos
\usepackage{microtype} 			% para melhorias de justificação

% pacotes adicionados
\usepackage{amsmath}
\usepackage{amssymb,amsfonts,amsthm}
\usepackage{setspace}
% ---
\usepackage{breqn}
\usepackage{amsmath}
% ---
% Pacotes adicionais, usados apenas no âmbito do Modelo Canônico do abnteX2
% ---
\usepackage{lipsum}				% para geração de dummy text
% ---
\usepackage{todonotes}
% ---
% Pacotes de citaçõeshttps://www.overleaf.com/project/5da65c39dd4248000106886a
% ---
\usepackage[brazilian,hyperpageref]{backref}	 % Paginas com as citações na bibl
\usepackage[alf]{abntex2cite}	% Citações padrão ABNT
\usepackage{listings}


\renewcommand{\bf}[1]{\mathbf{#1}}
\renewcommand{\rm}[1]{\mathrm{#1}}


\usepackage{cite}
\renewcommand\citeleft{[}
\renewcommand\citeright{]}

% --- 
% CONFIGURAÇÕES DE PACOTES
% --- 
\renewcommand{\imprimircapa}{
\thispagestyle{empty}

\vfill
 \begin{center}
    

    {\large\bfseries UNIVERSIDADE FEDERAL FLUMINENSE} \\
    
   
    {\large\bfseries INSTITUTO DE COMPUTAÇÃO}  \\ 

    \vspace*{1in}
    \begin{large} \bfseries João Pedro Abreu de Souza \end{large}\\[0.4in]

    \vspace*{4cm}
    \noindent \\
    
    \large\bfseries{ADICIONANDO CONTINUAÇÕES AO $\Pi$ FRAMEWORK} \\
    \vfill
    \large\bfseries{NITERÓI \\ 2019}
\end{center}

\normalsize


}


\renewcommand{\imprimirfolhaderosto}{

\begin{center}

    {\large JOÃO PEDRO ABREU DE SOUZA\\}
    \vspace{8cm}
    {\Large \textsc\textbf{{ADICIONANDO CONTINUAÇÕES AO $\Pi$ FRAMEWORK} }\\}
    \vspace{1cm}
    \hspace{.45\linewidth}
    \begin{minipage}{.50\linewidth}

            \textbf{Trabalho de Conclusão de Curso submetido à Universidade Federal Fluminense,  como requisito 
            necessário para obtenção do grau de Bacharel em Ciência da Computação }

           
    
    \end{minipage}

    \vspace{2cm}
    \vfill
    {\large Niterói, dezembro de 2019}
\end{center}

}


% ---
% Configurações do pacote backref
% Usado sem a opção hyperpageref de backref
\renewcommand{\backrefpagesname}{Citado na(s) página(s):~}
% Texto padrão antes do número das páginas
\renewcommand{\backref}{}
% Define os textos da citação
\renewcommand*{\backrefalt}[4]{
	\ifcase #1 %
		Nenhuma citação no texto.%
	\or
		Citado na página #2.%
	\else
		Citado #1 vezes nas páginas #2.%
	\fi}%
% ---
% ---
% Informações de dados para CAPA e FOLHA DE ROSTO
% ---

\titulo{Título do Trabalho}
\autor{Aluno}
\local{Brasil}
\data{2015, v-1.9.3}
\orientador{Orientador}
\coorientador{coorientador}
\instituicao{%
  Universidade Federal de Santa Catarina
  \par
  Departamento de Automação e Sistemas
  }
\tipotrabalho{Projeto de Fim de Curso}
% O preambulo deve conter o tipo do trabalho, o objetivo, 
% o nome da instituição e a área de concentração 
\preambulo{O preambulo deve conter o tipo do trabalho, o objetivo, o nome da instituição e a área de concentração }
% ---






% ---
% Configurações de aparência do PDF final

% alterando o aspecto da cor azul
\definecolor{blue}{RGB}{41,5,195}

% informações do PDF
\makeatletter
\hypersetup{
     	%pagebackref=true,
		colorlinks=true,       		% false: boxed links; true: colored links
    	linkcolor=blue,          	% color of internal links
    	citecolor=black,        		% color of links to bibliography
    	filecolor=magenta,      		% color of file links
		urlcolor=blue,
		bookmarksdepth=4
}



\makeatother
% --- 

% --- 
% Espaçamentos entre linhas e parágrafos 
% --- 

% O tamanho do parágrafo é dado por:
\setlength{\parindent}{1.3cm}

% Controle do espaçamento entre um parágrafo e outro:
\setlength{\parskip}{0.2cm}  % tente também \onelineskip

% ---
% compila o indice
% ---
\makeindex
% ---

% ----
% Início do documento
% ----
\begin{document}

% Seleciona o idioma do documento (conforme pacotes do babel)
%\selectlanguage{english}
\selectlanguage{brazil}

% Retira espaço extra obsoleto entre as frases.
\frenchspacing 

% ----------------------------------------------------------
% ELEMENTOS PRÉ-TEXTUAIS
% ----------------------------------------------------------
\pretextual

% ---
% Capa
% ---
\imprimircapa
% ---

% ---
% Folha de rosto
% (o * indica que haverá a ficha bibliográfica)
% ---
\imprimirfolhaderosto

% ---
% Inserir a ficha bibliografica
% ---

% Isto é um exemplo de Ficha Catalográfica, ou ``Dados internacionais de
% catalogação-na-publicação''. Você pode utilizar este modelo como referência. 
% Porém, provavelmente a biblioteca da sua universidade lhe fornecerá um PDF
% com a ficha catalográfica definitiva após a defesa do trabalho. Quando estiver
% com o documento, salve-o como PDF no diretório do seu projeto e substitua todo
% o conteúdo de implementação deste arquivo pelo comando abaixo:
%
% \begin{fichacatalografica}
%     \includepdf{fig_ficha_catalografica.pdf}
% \end{fichacatalografica}

\begin{fichacatalografica}
%	\sffamily
%	\vspace*{\fill}					% Posição vertical
%	\begin{center}					% Minipage Centralizado
%	\fbox{\begin{minipage}[c][8cm]{13.5cm}		% Largura
%	\small
%	\imprimirautor
%	%Sobrenome, Nome do autor
%	
%	\hspace{0.5cm} \imprimirtitulo  / \imprimirautor. --
%	\imprimirlocal, \imprimirdata-
%	
%	\hspace{0.5cm} \pageref{LastPage} p. : il. (algumas color.) ; 30 cm.\\
%	
%	\hspace{0.5cm} \imprimirorientadorRotulo~\imprimirorientador\\
%	
%	\hspace{0.5cm}
%	\parbox[t]{\textwidth}{\imprimirtipotrabalho~--~\imprimirinstituicao,
%	\imprimirdata.}\\
%	
%	\hspace{0.5cm}
%		1. Palavra-chave1.
%		2. Palavra-chave2.
%		2. Palavra-chave3.
%		I. Orientador.
%		II. Universidade xxx.
%		III. Faculdade de xxx.
%		IV. Título 			
%	\end{minipage}}
%	\end{center}
\end{fichacatalografica}
%% ---
% Inserir errata
% ---
\begin{errata}
Elemento opcional da \citeonline[4.2.1.2]{NBR14724:2011}. Exemplo:

\vspace{\onelineskip}

FERRIGNO, C. R. A. \textbf{Tratamento de neoplasias ósseas apendiculares com
reimplantação de enxerto ósseo autólogo autoclavado associado ao plasma
rico em plaquetas}: estudo crítico na cirurgia de preservação de membro em
cães. 2011. 128 f. Tese (Livre-Docência) - Faculdade de Medicina Veterinária e
Zootecnia, Universidade de São Paulo, São Paulo, 2011.

\begin{table}[htb]
\center
\footnotesize
\begin{tabular}{|p{1.4cm}|p{1cm}|p{3cm}|p{3cm}|}
  \hline
   \textbf{Folha} & \textbf{Linha}  & \textbf{Onde se lê}  & \textbf{Leia-se}  \\
    \hline
    1 & 10 & auto-conclavo & autoconclavo\\
   \hline
\end{tabular}
\end{table}

\end{errata}
% ---

% ---
% Inserir folha de aprovação
% ---

% Isto é um exemplo de Folha de aprovação, elemento obrigatório da NBR
% 14724/2011 (seção 4.2.1.3). Você pode utilizar este modelo até a aprovação
% do trabalho. Após isso, substitua todo o conteúdo deste arquivo por uma
% imagem da página assinada pela banca com o comando abaixo:
%
% \includepdf{folhadeaprovacao_final.pdf}
%
\begin{folhadeaprovacao}


\begin{center}


            {UNIVERSIDADE FEDERAL FLUMINENSE} \\
           

    \vspace{1.5cm}
                                    {JOÃO PEDRO ABREU DE SOUZA}\\
    \bfseries{}
\end{center}

Esta Monografia foi julgada adequada para a obten\c{c}\~{a}o do título  de Bacharel em Ciência da Computação, sendo aprovada em sua forma final  pela banca examinadora:

    \vspace{2.5cm}
    \assinatura{Orientador(a): Christiano Braga \\ Universidade Federal Fluminense - UFF}
    \vspace{3 cm}%\vfill

    \begin{center}
        Niteroi, 12 de dezembro de 2019
    \end{center}
  
\end{folhadeaprovacao}
%% ---
% Dedicatória
% ---
\begin{dedicatoria}
   \vspace*{\fill}
   \centering
   \noindent
   \textit{ Este trabalho é dedicado às crianças adultas que,\\
   quando pequenas, sonharam em se tornar cientistas.} \vspace*{\fill}
\end{dedicatoria}
% ---
% ---
% Agradecimentos
% ---
\begin{agradecimentos}
ou Acknowledgements, se em inglês \\
Opcional

\end{agradecimentos}
% ---
%% ---
% Epígrafe
% ---
\begin{epigrafe}
    \vspace*{\fill}
	\begin{flushright}
		\textit{``Não vos amoldeis às estruturas deste mundo, \\
		mas transformai-vos pela renovação da mente, \\
		a fim de distinguir qual é a vontade de Deus: \\
		o que é bom, o que Lhe é agradável, o que é perfeito.\\
		(Bíblia Sagrada, Romanos 12, 2)}
	\end{flushright}
\end{epigrafe}
% ---

% ---
% RESUMOS
% ---

% resumo em português
\setlength{\absparsep}{18pt} % ajusta o espaçamento dos parágrafos do resumo
\begin{resumo}
Esse trabalho se ocupa em estender o framework semântico $\Pi$-framework com uma construção básica (as continuações ilimitadas), uma maneira de captura-la (call/cc), e observar como expressar a semântica de construtos de mais alto nível como co-rotinas.


 \textbf{Palavras-chave}:.
\end{resumo}

% resumo em inglês
\begin{resumo}[Abstract]
 \begin{otherlanguage*}{english}
This work extends the semantic framework called $\Pi$-framework with a new basic construction(continuations), a way to make reification on it(call/cc), and look how code patterns that use this construction can express higher-level constructor(corotines).
\vspace{\onelineskip}
 
   \noindent 
   \textbf{Keywords}:Palavras Chaves.
 \end{otherlanguage*}
\end{resumo}

%% resumo em francês 
%\begin{resumo}[Résumé]
% \begin{otherlanguage*}{french}
%    Il s'agit d'un résumé en français.
% 
%   \textbf{Mots-clés}: latex. abntex. publication de textes.
% \end{otherlanguage*}
%\end{resumo}
%
%% resumo em espanhol
%\begin{resumo}[Resumen]
% \begin{otherlanguage*}{spanish}
%   Este es el resumen en español.
%  
%   \textbf{Palabras clave}: latex. abntex. publicación de textos.
% \end{otherlanguage*}
%\end{resumo}
% ---

% ---
% inserir lista de ilustrações
% ---
\pdfbookmark[0]{\listfigurename}{lof}
\listoffigures*
\cleardoublepage
%% ---

% ---
% inserir lista de tabelas
% ---
\pdfbookmark[0]{\listtablename}{lot}
\listoftables*
\cleardoublepage
% ---
% ---
% inserir lista de abreviaturas e siglas
% ---
\begin{siglas}
 \item Lista de Siglas
\end{siglas}
% ---

% ---
% inserir lista de símbolos
% ---
%\begin{simbolos}
%  \item[$ CO_2 $] Dióxido de Carbono
%  \item[$C_{3+}$] Hidrocarbonetos com três ou mais carbonos
%\end{simbolos}
% ---





% ---
% inserir o sumario
%% ---
\pdfbookmark[0]{\contentsname}{toc}
\tableofcontents*
\cleardoublepage
%% ---



% ----------------------------------------------------------
% ELEMENTOS TEXTUAIS
% ----------------------------------------------------------
\textual

\chapter{Introdução}

%%%% Sobre o que trata este trabalho?
%%%% Por que este trabalho foi feito? (A importância de continuações como mecanismo de controle. Pi automato como um modelo simples para ensino de compiladores feito no ic/uff no qual continuacoes sao explicitas.) 
%%%% Como foi feito?

Esse trabalho trata da adição de continuações à um Framework semântico para ensino de compiladores chamado $\Pi$ Framework. Continuações apresentam um controle de fluxo não-local. 

Continuações surgiram na linguagem scheme através da função call-with-current-continuation, ou \texttt{call/cc}. Ela apresenta importância prática e teórica, sendo possível utiliza-la para implementar \texttt{amb}, corotinas, continuações delimitadas e exceções, além de, no isomorfismo de Curry-Howard, seu tipo ser equivalente a lei de Pierce.

O $\Pi$ Framework é um modelo criado por Christiano Braga para ensino da disciplina de compiladores. Este framework contém elementos básicos para descrever a semântica de reescrita de linguagens de programação, como estruturas de repetição, decisão, procedimentos e funções recursivas.

Esse capitulo dará uma breve exposição dos recursos tratados ou utilizados para a produção desta monografia. O Capítulo \ref{cap2} definirá em maiores detalhes o estado atual da maquina abstrata utilizada para dar semântica a linguagens de programação. O capítulo \ref{cap3} adicionará ao capítulo \ref{cap2} as contribuições do autor a esta maquina. O capitulo 4 exemplificará como as novas primitivas podem ser utilizadas para descrever a semântica de construções como retorno de funções e corotinas. Finalmente, o capítulo 5 citará os trabalhos que influenciaram a presente monografia ou atacaram o mesmo problema de semântica formal.

\section{Racket}
Racket\footnote{https://racket-lang.org/}, inicialmente chamada de plt-scheme, é uma linguagem baseada em lisp criada para ser uma linguagem para desenvolver linguagens. Multi-paradigma, priorizando estruturas de dados imutáveis, com um sistema de meta-programação baseado em macros higiênicas. Com extensa facilidade para realizar pattern matching, Racket foi a linguagem escolhida para desenvolver esse trabalho. O principal procedimento desenvolvido pelo autor, a saber, a função de transição do sistema, faz uso extenso do pattern matching, permitindo que as regras de transição sejam codificadas sem maior esforço, de uma maneira declarativa.

Ex:
    Dada uma estrutura \texttt{add}, contendo dois membros númericos, sua avaliação dada através da regra
    $$(\texttt{add a b)} \xrightarrow{} a+b  $$
    Pode ser codificada como a função \texttt{evalExpression} dada por
    
    \begin{verbatim}
    (define (evalExpression exp)
    
            (match exp
            
            [(add a b) (+ a b)]))
    \end{verbatim}
    
\section{Pi Automato}

\section{Continuações}

\section{Call/cc}
%faria sentido ter uma seção explicando sobre o racket
\subsection{Call/cc em Racket}
\subsubsection{Ret}
\subsubsection{Call/cc}
\subsubsection{Co-rotinas}
\chapter{\textit{$\Pi$ Autômato}}
Esse capítulo tratará do $\Pi$ Autômato antes da introdução das continuações, portanto, do seu estado atual.
\section{Definição}
O $\Pi$ Autômato é uma máquina abstrata, contendo duas pilhas (a saber, pilha de valores, que armazena principalmente valores intermediários de uma computação; e pilha de controle, que armazena o resto dos comandos e/ou declarações a partir do ponto atual de execução) e dois mapas finitos (a saber, o "ambiente", que controla os bindings de constantes e a associação entre identificadores e índices (chamados "localizações") do segundo hash, chamado de "memória", que mapeia as localizações para números, booleanos e abstrações.
\subsection{PI IR}
\subsubsection{Expressões}
\subsubsection{Comandos}
\subsubsection{Declarações}

\subsection{Exemplos de PI IR}
\subsubsection{Expressões}
\subsubsection{Comandos}
\subsubsection{Declarações}\label{declarações}
%%% TEXTO representando codigo deve ser formatado em tt.
%%% TEXTO representando elementos matematicos deve estar entre $$. Se for uma palavra, dentro de mathit.

\chapter{$\Pi$ Autômato com continuações}
Esse capítulo mostrará como as continuações são expressas no formalismo $\Pi$ Framework.
A seção \ref{secao3.1} discorrerá qualitativamente sobre as construções $\Pi$ IR necessárias para implementar a semântica de continuações. A seção \ref{secao3.2} discorrerá sobre corotinas, que não exigem construções novas no $\Pi$ IR, porém se utilizam das construções descritas na seção \ref{secao3.1} para prover a sua semântica. Finalmente, a \ref{primeira-cont} irá apresentar as transições, no $\Pi$ Framework, das estruturas descritas na seção \ref{secao3.1}.

\section{Primitivas $\Pi$ IR para continuações}\label{secao3.1}

\subsection{\texttt{Cont}}

Todo termo em $\Pi$ IR que é uma expressão (e toda avaliação de expressão do topo da pilha de controle no $\Pi$ autômato) possui uma continuação implícita, visto que, para terminar a execução, o $\Pi$ autômato precisa da pilha de valores vazia e toda expressão termina por colocar um valor no topo da pilha de valores. Dessa forma, são necessárias mais transições para completar a computação. Essas transições são a continuação implícita. 

Com a possibilidade de capturar uma continuação (através do $\mathit{call/cc}$ explicado na seção \ref{call/cc}), surge a necessidade de incorporá-la como valor, visto que agora pode, por exemplo, ser atribuída a uma variável como uma expressão qualquer. Toda expressão é um valor primitivo, imediato, como um número ou um booleano, ou estruturas complexas, como adição de expressões ou \texttt{or} de expressões. Um valor primitivo, imediato, da origem a uma transição que o leva do topo da pilha de controle para o topo da pilha de valores. Uma expressão complexa exige sua desconstrução, avaliação de suas sub-expressões, e por fim a avaliação final de seu valor. Por exemplo, para avaliar $$2+(2+2)$$ sob um contexto arbitrário, que é uma expressão complexa, deve-se desconstruí-la em suas sub-expressões ($2$ e $(2+2)$, obter o valor de cada sub-expressão ($2$ e $4$), e por fim, avaliar a expressão final com o resultado $6$. Uma continuação, embora seja uma expressão, não se enquadra na categoria de expressões aritméticas nem booleanas. Exige para si portanto uma categoria nova. Nessa categoria nova, as continuações, ao serem avaliadas, não necessitam de analise de sub-expressões, portanto se assemelham em comportamento aos números e booleanos. Como tal, se estiverem no topo da pilha de controle, são apenas transferidas para o topo da pilha de valores, como indicado pela seção \ref{primeira-cont}.

Todo valor primitivo possui operações associadas a este, como os números, que podem ser somados, subtraídos, multiplicados e divididos; Os valores booleanos, que podem ser negados, aplicados "ou", "e". Com continuações não é diferente. A operação particular da continuação é a invocação com uma única expressão. Essa expressão deve se tornar um valor, e o contexto de execução representado na continuação é restaurado, com este valor ocorrendo em substituição ao foco da continuação supracitada. Essa propriedade é exposta na segunda equação sobre continuações\ref{segunda-cont}.

\subsection{\texttt{CallActualsf}}

O \texttt{CallActualsf} representa uma avaliação estrita de uma função de aridade arbitrária. A avaliação estrita é exposta na primeira equação\ref{primeira-callf} associada a \texttt{CallActualsf}
Toda função, ao contrario de um procedimento, necessita retornar um valor.
A expressão que contém a chamada da função, portanto, precisa ser relembrada.
Essa "lembrança" aparece na forma da continuação desta chamada. Essa "lembrança" é exposta na segunda equação\ref{segunda-callf} associada a \texttt{callActualsf}.

\subsection{\texttt{Call/cc}\label{call/cc}}

Embora continuações existam necessariamente na avaliação de uma expressão, se não puderem ser trazidas à avaliação de um programa explicitamente, só seriam utilizadas para controlar retorno de funções, como dito acima. Com o \texttt{$\mathit{call/cc}$}, a continuação associada a chamada do \texttt{$\mathit{call/cc}$} pode ser reificada (trazida explicitamente ao programa) e manipulada de forma geral por uma função. Essa propriedade é exposta na única equação associada a $\mathit{call/cc}$\ref{primeira-callcc}.

\section{Codificando Corotinas}\label{secao3.2}

Para um exemplo do uso de corotinas utilizando o $\Pi$ Framework para resolver o problema de consumidor-produtor, veja~\cite{piLibSemantic}.

Uma corotina que não chama outras corotinas não se diferencia muito de uma subrotina. Sua semântica especifica é exibida quando da interação com outras
corotinas. Não são necessárias, dado as transições para continuações e $\mathit{call/cc}$, quaisquer transições especiais para a definição de corotinas. Elas porém se caracterizam por dois padrões descritos a seguir.

\subsection{Funções recebendo continuações}

%%% ESPAÇO ANTES DO PARENTESIS

%%% formatacao de variaveis

%%% pi-ir 

Toda corotina (em alto nível) é uma função (em baixo nível) que recebe uma continuação como argumento.\footnote{Atualmente no $\Pi$ IR são aceitas apenas corotinas que interagem em pares, por isso a restrição de um único argumento (no caso, dadas corotinas $\mathit{f1}$ e $\mathit{f2}$, o argumento de $\mathit{f1}$ representa a continuação de $\mathit{f2}$ e o argumento de $\mathit{f2}$ representa a continuação de $\mathit{f1}$). Para maior interação, trabalho posterior incluirá o tratamento de mais argumentos.} 
Esse argumento representa o estado atual (ambiente e ponto do código em execução) da outra corotina, com a qual a primeira interage. Ao longo da execução de uma corotina, ela deve obedecer às seguintes condições\label{rules-corotines}:
\begin{itemize}
    \item quando for chamada por um código de fora da dupla de corotinas (se existem duas corotinas $\mathit{f1}$ e $\mathit{f2}$, e uma função main, que não é corotina, main será o dito "código de fora") deverá ser chamada com a abstração que representa a outra corotina (main chamaria $\mathit{f1}$, passando $\mathit{f2}$ como argumento no exemplo);
    \item quando for chamar a outra corotina ($\mathit{f1}$ invocar $\mathit{f2}$, no exemplo), deve atualizar seu argumento com a continuação nova da outra corotina (se $\mathit{f1}$ possui o argumento $\mathit{k1}$, a invocação de $\mathit{f2}$ se dará como $Assign(Idt(k1), call/cc(k1)) )$.
\end{itemize}

Se uma função f obedecer as propriedades acima\ref{rules-corotines}, \texttt{f} poderá ser usada como corotina.

\subsection{Explicação da segunda regra de corotina\ref{rules-corotines}}
O que acontece nessa atribuição é uma manipulação sofisticada de continuações. Sendo \texttt{$\mathit{k1}$} uma continuação de $g$, $\mathit{call/cc}$ capturará a continuação atual de $f$, e a entregará a $\mathit{k1}$, efetivamente invocando $g$, do ponto que g estava da última vez que $f$ e $g$ interagiram, com a continuação atual de $f$. Quando $g$ retornar dessa interação, invocando o mesmo comando (\textit{mutatis mutandis}, pois o parâmetro de continuação de $g$ pode ter um nome diferente de $\mathit{k1}$), sua própria continuação será capturada pelo $\mathit{call/cc}$ e entregue a continuação atual de $f$. A continuação atual de $g$ será então atribuída a $\mathit{k1}$. Nesse contexto, $f$ terá interagido com $g$, continuará sua execução do ponto que retornou, e tem em seu parâmetro o ponto exato em que $g$ deixou de executar anteriormente.

\section{Transições}

%%% RELEMBRE A NOTACAO. RELEMBRE AS OUTRAS PRIMITIVAS NECESSÁRIAS AO 
%%% ENTENDIMENTO DAS PRIMITIVAS PARA CONTINUACOES.
%%% DECLARE AS VARIAVEIS.

O estado do $\Pi$ autômato é representado, em notação da função $\delta$, tipicamente utilizada em teoria de autômatos, onde a linguagem é $(\Pi \mathit{IR})^*$ e o estado é formado pela Pilha de valores, Ambiente e Memória. A função de transição tem a seguinte forma geral: 
$$
\delta(C, V, E, S)
$$ onde $C$ é a pilha de controle, $V$ é a pilha de valores, $E$ é o ambiente e $S$ é a memória.

As variáveis $C$ e $V$ são pilhas, portanto, ou estão vazias $[]$ ou possuem um valor em seu topo e outra pilha como cauda

$$X :: XS$$

E e S são mapas finitos, portanto ou estão vazios $$\{\}$$ ou são a extensão de outro mapa finito por uma associação de chave e valor $$H[k\rightarrow{}v].$$

Transições de estados são representadas como $$E1 \rightarrow{} E2$$ com $E1$ e $E2$ sendo o estado corrente e o próximo estado, respectivamente.

%%% JOÃO FALANDO
%%% As explicações aqui foram dadas na seção 3.1 inteira.
%%% Eu deveria mover aquelas explicações para essa seção?
%%% Ou repeti-las de outra maneira, sob algum outro foco?

%%% Intercale as explicacoes com as equacoes.

\subsection{\textbf{Continuações}}


%%%% Normalize entre Inglês e Português.
Como uma continuação (dada pelo construtor $Cont(Environment,ValueStack,ControlStack)$ com Environment, ValueStack e ControlStack definidos com os tipos acima mencionados na seção \ref{primeira-cont} é uma expressão, deve ser transportada, quando for o termo corrente (topo da pilha de controle), para a pilha de valores.
%\begin{multline}
%\delta(Cont(Env,Val,Con) :: C, V, E, S) \\
%\rightarrow{} \\
%\delta(C,Cont(Env,Val,Con)::V, E, S)         \\
%\end{multline}

$$
\delta(Cont(Env,Val,Con) :: C, V, E, S) 
\rightarrow{}
\delta(C,Cont(Env,Val,Con)::V, E, S)     
$$\label{primeira-cont}

Como uma continuação representa o contexto de uma expressão, sua invocação supre esse contexto (essencialmente eliminando o contexto atual) com o valor resultante da avaliação da expressão que lhe foi passada por parâmetro.

$$
\delta(\#CALLF :: C, Actual :: Cont(Env,Val,Con) :: V, E, S)
\rightarrow{}
\delta(Con,Actual::Val, Env, S)
$$\label{segunda-cont}


\subsection{\textbf{CallActualsf}}
Uma chamada de função é uma expressão resultante da avaliação de um corpo parametrizável de comandos terminados (logicamente, não fisicamente) com um retorno. Quando esse retorno ocorre, é necessário que a execução substitua, no contexto da chamada original da função, o valor da expressão que serviu de parâmetro para o retorno. Esse contexto é então representado pela continuação guardada na pilha de valores. O desdobramento do CallActualsf segue o padrão bastante presente no $\Pi$ Framework de avaliar os argumentos de forma estrita, da esquerda para a direita.

$$
\delta(CallActualsf(Id,Actuals)::C, V,E,S)
\rightarrow{}
\delta(Id::Actuals::\#CALLF::C, Cont(E,V,C)::V, E,S)
$$\label{primeira-callf}

Uma vez a indução na avaliação do corpo da função e dos argumentos realizada, é necessário tornar a sua execução um "comando", e isto é feito criando-se um bloco contendo o corpo da função como comando, e os \textit{bindings} dos parâmetros formais aos atuais como variáveis locais.

$$
\delta(\#CALLF :: C, Actuals ... :: AbsFormals(Formals,Block) :: V, E,S)
\rightarrow{}
\delta(BlkCommandDec(casa(Formals,Actuals),Block)::C,V,E,S)
$$\label{segunda-callf}
$$casa(Par(a),e) = Bind(a,Ref(e))$$
$$casa(For(a,b),e1::r) = Dec(Bind(a,Ref(e1)),casa(b,r))$$
onde $\mathit{casa}$ fornece os bindings 


%%% Normalize ingles e portugues
%%% formate o texto dos construtores
dos parâmetros formais com os valores passados na chamada da função, onde Par representa um parâmetro formal da função, For sequencia parâmetros formais, Dec sequencia declarações, Ref obtém uma localização limpa para associa-la, em memória, a uma expressão, e Bind associa, no ambiente, um identificador a uma localização. Todos esses construtores, bem como sua semântica, são explicados na seção \ref{declarações}

%%% João falando
%%% se essa estrutura esta correta, agora devo descrever o capitulo 2, para preencher essa referência, e expor o pi automato, nos mesmos moldes de qualidade desse capitulo, certo?

%%% O QUE É 'casa'?


\subsection{\textbf{Call/cc}}

Como toda expressão, Call/cc possui uma continuação. Ele portanto captura a sua própria continuação, e a utiliza como argumento do seu próprio argumento. Deve ficar claro que a continuação de Call/cc não é eliminada quando ocorre a chamada do seu argumento, portanto se o seu argumento ignora a continuação e só retorna 3, por exemplo, o efeito gerado é o mesmo que substituir o Call/cc por 3 no contexto do $\Pi$ automata

$$
\delta(Call/cc(Func) :: C, V, E, S)
\rightarrow{}
\delta(Callf(Func, Cont(E,V,C)) :: C,V, E, S)
$$\label{primeira-callcc}


\chapter{Exemplos de uso de Continuações em Pi Automato}\label{cap4}

\section{Ret}
\subsection{Racket}
\subsection{PI IR}
\subsection{Transições completas}
\section{Call/cc}
\subsection{Racket}
\subsection{PI IR}
\subsection{Transições completas}
\section{Corotinas}
\subsection{Racket}
\subsection{PI IR}
\subsection{Transições completas}

\chapter{Trabalhos Relacionados}
\section{CPS}
\section{Component-based semantics}
\section{K framework}
\chapter{Conclusão}\label{cap6}




% ----------------------------------------------------------
% Finaliza a parte no bookmark do PDF
% para que se inicie o bookmark na raiz
% e adiciona espaço de parte no Sumário
% ----------------------------------------------------------
\phantompart

% ---
% Conclusão
% ---
%\chapter{Conclusão}
% ---

% ----------------------------------------------------------
% ELEMENTOS PÓS-TEXTUAIS
% ----------------------------------------------------------
\postextual
% ----------------------------------------------------------

% ----------------------------------------------------------
% Referências bibliográficas
% ----------------------------------------------------------
\bibliography{TCC.bib}

% ----------------------------------------------------------
% Glossário
% ----------------------------------------------------------
%
% Consulte o manual da classe abntex2 para orientações sobre o glossário.
%
%\glossary

%% ----------------------------------------------------------
% Apêndices
% ----------------------------------------------------------

% ---
% Inicia os apêndices
% ---
\begin{apendicesenv}

% Imprime uma página indicando o início dos apêndices
\partapendices

% ----------------------------------------------------------
\chapter{Quisque libero justo}
% ----------------------------------------------------------

\lipsum[50]

% ----------------------------------------------------------
\chapter{Nullam elementum urna vel imperdiet sodales elit ipsum pharetra ligula
ac pretium ante justo a nulla curabitur tristique arcu eu metus}
% ----------------------------------------------------------
\lipsum[55-57]

\end{apendicesenv}
% ---

%% ----------------------------------------------------------
% Anexos
% ----------------------------------------------------------

% ---
% Inicia os anexos
% ---
\begin{anexosenv}

% Imprime uma página indicando o início dos anexos
\partanexos

% ---
\chapter{Morbi ultrices rutrum lorem.}
% ---
\lipsum[30]

% ---
\chapter{Cras non urna sed feugiat cum sociis natoque penatibus et magnis dis
parturient montes nascetur ridiculus mus}
% ---

\lipsum[31]

% ---
\chapter{Fusce facilisis lacinia dui}
% ---

\lipsum[32]

\end{anexosenv}

%---------------------------------------------------------------------
% INDICE REMISSIVO
%---------------------------------------------------------------------
\phantompart
\printindex
%---------------------------------------------------------------------

\end{document}
